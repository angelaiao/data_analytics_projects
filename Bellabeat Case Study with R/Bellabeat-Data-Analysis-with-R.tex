% Options for packages loaded elsewhere
\PassOptionsToPackage{unicode}{hyperref}
\PassOptionsToPackage{hyphens}{url}
%
\documentclass[
]{article}
\usepackage{amsmath,amssymb}
\usepackage{lmodern}
\usepackage{ifxetex,ifluatex}
\ifnum 0\ifxetex 1\fi\ifluatex 1\fi=0 % if pdftex
  \usepackage[T1]{fontenc}
  \usepackage[utf8]{inputenc}
  \usepackage{textcomp} % provide euro and other symbols
\else % if luatex or xetex
  \usepackage{unicode-math}
  \defaultfontfeatures{Scale=MatchLowercase}
  \defaultfontfeatures[\rmfamily]{Ligatures=TeX,Scale=1}
\fi
% Use upquote if available, for straight quotes in verbatim environments
\IfFileExists{upquote.sty}{\usepackage{upquote}}{}
\IfFileExists{microtype.sty}{% use microtype if available
  \usepackage[]{microtype}
  \UseMicrotypeSet[protrusion]{basicmath} % disable protrusion for tt fonts
}{}
\makeatletter
\@ifundefined{KOMAClassName}{% if non-KOMA class
  \IfFileExists{parskip.sty}{%
    \usepackage{parskip}
  }{% else
    \setlength{\parindent}{0pt}
    \setlength{\parskip}{6pt plus 2pt minus 1pt}}
}{% if KOMA class
  \KOMAoptions{parskip=half}}
\makeatother
\usepackage{xcolor}
\IfFileExists{xurl.sty}{\usepackage{xurl}}{} % add URL line breaks if available
\IfFileExists{bookmark.sty}{\usepackage{bookmark}}{\usepackage{hyperref}}
\hypersetup{
  pdftitle={How Can a Wellness Technology Company Play It Smart?},
  pdfauthor={Angela Iao},
  hidelinks,
  pdfcreator={LaTeX via pandoc}}
\urlstyle{same} % disable monospaced font for URLs
\usepackage[margin=1in]{geometry}
\usepackage{color}
\usepackage{fancyvrb}
\newcommand{\VerbBar}{|}
\newcommand{\VERB}{\Verb[commandchars=\\\{\}]}
\DefineVerbatimEnvironment{Highlighting}{Verbatim}{commandchars=\\\{\}}
% Add ',fontsize=\small' for more characters per line
\usepackage{framed}
\definecolor{shadecolor}{RGB}{248,248,248}
\newenvironment{Shaded}{\begin{snugshade}}{\end{snugshade}}
\newcommand{\AlertTok}[1]{\textcolor[rgb]{0.94,0.16,0.16}{#1}}
\newcommand{\AnnotationTok}[1]{\textcolor[rgb]{0.56,0.35,0.01}{\textbf{\textit{#1}}}}
\newcommand{\AttributeTok}[1]{\textcolor[rgb]{0.77,0.63,0.00}{#1}}
\newcommand{\BaseNTok}[1]{\textcolor[rgb]{0.00,0.00,0.81}{#1}}
\newcommand{\BuiltInTok}[1]{#1}
\newcommand{\CharTok}[1]{\textcolor[rgb]{0.31,0.60,0.02}{#1}}
\newcommand{\CommentTok}[1]{\textcolor[rgb]{0.56,0.35,0.01}{\textit{#1}}}
\newcommand{\CommentVarTok}[1]{\textcolor[rgb]{0.56,0.35,0.01}{\textbf{\textit{#1}}}}
\newcommand{\ConstantTok}[1]{\textcolor[rgb]{0.00,0.00,0.00}{#1}}
\newcommand{\ControlFlowTok}[1]{\textcolor[rgb]{0.13,0.29,0.53}{\textbf{#1}}}
\newcommand{\DataTypeTok}[1]{\textcolor[rgb]{0.13,0.29,0.53}{#1}}
\newcommand{\DecValTok}[1]{\textcolor[rgb]{0.00,0.00,0.81}{#1}}
\newcommand{\DocumentationTok}[1]{\textcolor[rgb]{0.56,0.35,0.01}{\textbf{\textit{#1}}}}
\newcommand{\ErrorTok}[1]{\textcolor[rgb]{0.64,0.00,0.00}{\textbf{#1}}}
\newcommand{\ExtensionTok}[1]{#1}
\newcommand{\FloatTok}[1]{\textcolor[rgb]{0.00,0.00,0.81}{#1}}
\newcommand{\FunctionTok}[1]{\textcolor[rgb]{0.00,0.00,0.00}{#1}}
\newcommand{\ImportTok}[1]{#1}
\newcommand{\InformationTok}[1]{\textcolor[rgb]{0.56,0.35,0.01}{\textbf{\textit{#1}}}}
\newcommand{\KeywordTok}[1]{\textcolor[rgb]{0.13,0.29,0.53}{\textbf{#1}}}
\newcommand{\NormalTok}[1]{#1}
\newcommand{\OperatorTok}[1]{\textcolor[rgb]{0.81,0.36,0.00}{\textbf{#1}}}
\newcommand{\OtherTok}[1]{\textcolor[rgb]{0.56,0.35,0.01}{#1}}
\newcommand{\PreprocessorTok}[1]{\textcolor[rgb]{0.56,0.35,0.01}{\textit{#1}}}
\newcommand{\RegionMarkerTok}[1]{#1}
\newcommand{\SpecialCharTok}[1]{\textcolor[rgb]{0.00,0.00,0.00}{#1}}
\newcommand{\SpecialStringTok}[1]{\textcolor[rgb]{0.31,0.60,0.02}{#1}}
\newcommand{\StringTok}[1]{\textcolor[rgb]{0.31,0.60,0.02}{#1}}
\newcommand{\VariableTok}[1]{\textcolor[rgb]{0.00,0.00,0.00}{#1}}
\newcommand{\VerbatimStringTok}[1]{\textcolor[rgb]{0.31,0.60,0.02}{#1}}
\newcommand{\WarningTok}[1]{\textcolor[rgb]{0.56,0.35,0.01}{\textbf{\textit{#1}}}}
\usepackage{longtable,booktabs,array}
\usepackage{calc} % for calculating minipage widths
% Correct order of tables after \paragraph or \subparagraph
\usepackage{etoolbox}
\makeatletter
\patchcmd\longtable{\par}{\if@noskipsec\mbox{}\fi\par}{}{}
\makeatother
% Allow footnotes in longtable head/foot
\IfFileExists{footnotehyper.sty}{\usepackage{footnotehyper}}{\usepackage{footnote}}
\makesavenoteenv{longtable}
\usepackage{graphicx}
\makeatletter
\def\maxwidth{\ifdim\Gin@nat@width>\linewidth\linewidth\else\Gin@nat@width\fi}
\def\maxheight{\ifdim\Gin@nat@height>\textheight\textheight\else\Gin@nat@height\fi}
\makeatother
% Scale images if necessary, so that they will not overflow the page
% margins by default, and it is still possible to overwrite the defaults
% using explicit options in \includegraphics[width, height, ...]{}
\setkeys{Gin}{width=\maxwidth,height=\maxheight,keepaspectratio}
% Set default figure placement to htbp
\makeatletter
\def\fps@figure{htbp}
\makeatother
\setlength{\emergencystretch}{3em} % prevent overfull lines
\providecommand{\tightlist}{%
  \setlength{\itemsep}{0pt}\setlength{\parskip}{0pt}}
\setcounter{secnumdepth}{-\maxdimen} % remove section numbering
\ifluatex
  \usepackage{selnolig}  % disable illegal ligatures
\fi

\title{How Can a Wellness Technology Company Play It Smart?}
\author{Angela Iao}
\date{Last updated Aug 2022}

\begin{document}
\maketitle

\hypertarget{scenario}{%
\section{Scenario}\label{scenario}}

I am a junior data analyst working on the marketing analyst team at
Bellabeat, a high-tech manufacturer of health-focused products for
women. Bellabeat is a successful small company, but they have the
potential to become a larger player in the global smart device market.
Urška Sršen, co-founder and Chief Creative Officer of Bellabeat,
believes that analyzing smart device fitness data could help unlock new
growth opportunities for the company. You have been asked to focus on
one of Bellabeat's products and analyze smart device data to gain
insight into how consumers are using their smart devices. The insights
you discover will then help guide the marketing strategy for the
company. You will present your analysis to the Bellabeat executive team
along with your high-level recommendations for Bellabeat's marketing
strategy.

\hypertarget{products}{%
\subsection{Products}\label{products}}

\begin{itemize}
\tightlist
\item
  \textbf{Bellabeat app}: The Bellabeat app provides users with health
  data related to their activity, sleep, stress, menstrual cycle, and
  mindfulness habits. This data can help users better understand their
  current habits and make healthy decisions. The Bellabeat app connects
  to their line of smart wellness products.
\item
  \textbf{Leaf}: Bellabeat's classic wellness tracker can be worn as a
  bracelet, necklace, or clip. The Leaf tracker connects to the
  Bellabeat app to track activity, sleep, and stress.
\item
  \textbf{Time}: This wellness watch combines the timeless look of a
  classic timepiece with smart technology to track user activity, sleep,
  and stress. The Time watch connects to the Bellabeat app to provide
  you with insights into your daily wellness.
\item
  \textbf{Spring}: This is a water bottle that tracks daily water intake
  using smart technology to ensure that you are appropriately hydrated
  throughout the day. The Spring bottle connects to the Bellabeat app to
  track your hydration levels.
\item
  \textbf{Bellabeat membership}: Bellabeat also offers a
  subscription-based membership program for users. Membership gives
  users 24/7 access to fully personalized guidance on nutrition,
  activity, sleep, health and beauty, and mindfulness-based on their
  lifestyle and goals.
\end{itemize}

\hypertarget{about-bellabeat}{%
\section{About Bellabeat}\label{about-bellabeat}}

Urška Sršen and Sando Mur founded Bellabeat, a high-tech company that
manufactures health-focused smart products. Sršen used her background as
an artist to develop beautifully designed technology that informs and
inspires women around the world. Collecting data on activity, sleep,
stress, and reproductive health has allowed Bellabeat to empower women
with knowledge about their own health and habits. Since it was founded
in 2013, Bellabeat has grown rapidly and quickly positioned itself as a
tech-driven wellness company for women. By 2016, Bellabeat had opened
offices around the world and launched multiple products. Bellabeat
products became available through a growing number of online retailers
in addition to their own e-commerce channel on their website. The
company has invested in traditional advertising media, such as radio,
out-of-home billboards, print, and television, but focuses on digital
marketing extensively. Bellabeat invests year-round in Google Search,
maintains active Facebook and Instagram pages, and consistently engages
consumers on Twitter. Additionally, Bellabeat runs video ads on Youtube
and display ads on the Google Display Network to support campaigns
around key marketing dates.

Sršen knows that an analysis of Bellabeat's available consumer data
would reveal more opportunities for growth. She has asked the marketing
analytics team to focus on a Bellabeat product and analyze smart device
usage data in order to gain insight into how people are already using
their smart devices. Then, using this information, she would like
high-level recommendations for how these trends can inform Bellabeat
marketing strategy.

\hypertarget{ask}{%
\section{Ask}\label{ask}}

\hypertarget{business-task}{%
\subsection{Business Task}\label{business-task}}

The \textbf{key business task} is to analyze smart device usage data to
gain insights into how customers use smart devices. I will apply those
insights and make data-driven recommendations for the Bellabeat
marketing analytics team.

\hypertarget{stakeholders}{%
\subsubsection{Stakeholders}\label{stakeholders}}

\begin{itemize}
\tightlist
\item
  \textbf{Urška Sršen}: Bellabeat's co-founder and Chief Creative
  Officer
\item
  \textbf{Sandro Mur}: Mathematician and Bellabeat co-founder; key
  member of the Bellabeat executive team
\item
  \textbf{Bellabeat marketing analytics team}: A team of data analysts
  responsible for collecting, analyzing, and reporting data that helps
  guide Bellabeat's marketing strategy. You joined this team six months
  ago and have been busy learning about Bellabeat's mission and business
  goals --- as well as how you, as a junior data analyst, can help
  Bellabeat achieve them.
\end{itemize}

\hypertarget{prepare}{%
\section{Prepare}\label{prepare}}

\hypertarget{data-source}{%
\subsection{Data Source}\label{data-source}}

\textbf{Data source}: \url{https://www.kaggle.com/arashnic/fitbit} (CC0:
Public Domain, dataset made available through Mobius)

The dataset, FitBit Fitness Tracker Data, consists of 18 CVS files that
are organized in a long format. The dataset contains personal fitness
tracker data from 30 Fitbit users, generated from a distributed survey
via Amazon Mechanical Turk between March 12, 2016, to May 12, 2016. 30
participants consented to the submission of personal tracker data,
including minute-level output for physical activity, heart rate, and
sleep monitoring. It includes information about daily activity, steps,
and sleep data that can be used to explore users' habits.

\hypertarget{data-integrity}{%
\subsubsection{Data Integrity}\label{data-integrity}}

Please take into consideration that this dataset has various limitations
and may not provide reliable insights. Highly recommend finding more
accurate and reliable data sources.

\hypertarget{limitations}{%
\paragraph{Limitations}\label{limitations}}

{Reliability} \textbar{} \emph{Low} \textbar{} The dataset is incomplete
due to the small sample size (n = 30). The data source does not mention
whether the sample is selected at random, potentially sampling bias.

{Original} \textbar{} \emph{Low} \textbar{} The dataset is generated
from a survey via Amazon Mechanical Turk, so it may not be accurate.

{Comprehensive} \textbar{} \emph{Low} \textbar{} The data has sufficient
data, however, the parameters and attributes are undefined and unclear.

{Current} \textbar{} \emph{Low} \textbar{} The survey was done 5 years
ago in 2016, not representative of present times. The dataset is
outdated.

\hypertarget{process}{%
\section{Process}\label{process}}

\hypertarget{dataset-parameters}{%
\subsection{Dataset Parameters}\label{dataset-parameters}}

There are 18 CSV files in this dataset:

\emph{{Weight}}

\begin{enumerate}
\def\labelenumi{\arabic{enumi}.}
\tightlist
\item
  \textbf{weightloginfo} → Id, Date, WeightKg, WeightPounds, Fat, BMI,
  IsManualReport, LogID
\end{enumerate}

\emph{{METs (Metabolic Equivalent of Tasks}}

\begin{enumerate}
\def\labelenumi{\arabic{enumi}.}
\setcounter{enumi}{1}
\tightlist
\item
  \textbf{minuteMETsNarrow} → id, ActivityMinute, METs
\end{enumerate}

\emph{{Steps}}

\begin{enumerate}
\def\labelenumi{\arabic{enumi}.}
\setcounter{enumi}{2}
\tightlist
\item
  \textbf{minuteStepsWide} → id, ActivityHour, StepsNN
\item
  \textbf{minuteStepsNarrow} → id, ActivityMinute, Steps
\item
  \textbf{hourlySteps} → id, ActivityHour, StepTotal
\item
  \textbf{dailySteps} → id, ActivityDay, StepTotal
\end{enumerate}

\emph{{Sleep}}

\begin{enumerate}
\def\labelenumi{\arabic{enumi}.}
\setcounter{enumi}{6}
\tightlist
\item
  \textbf{sleepDay} → id, SleepDay, TotalSleepRecords,
  TotalMinutesAsleep, TotalTimeInBed
\item
  \textbf{minuteSleep} → id, date, value, logid
\end{enumerate}

\emph{{Intensities}}

\begin{enumerate}
\def\labelenumi{\arabic{enumi}.}
\setcounter{enumi}{8}
\tightlist
\item
  \textbf{minuteIntensitiesWide} → id, ActivityHour, IntensityNN
\item
  \textbf{minuteIntensitiesNarrow} → id, ActivityHour, Intensity
\item
  \textbf{hourlyIntensities8} → id, ActivityHour, TotalIntensity,
  AverageIntensity
\item
  \textbf{dailyIntensities} → id, ActivityDay, Sedentary Minutes,
  LightlyActiveMinutes, FairlyActiveMinutes, VeryActiveMinutes,
  SedentaryActiveDistance, LightActiveDistance,
  ModeratelyActiveDistance, VeryActiveDistance
\end{enumerate}

\emph{{Calories}}

\begin{enumerate}
\def\labelenumi{\arabic{enumi}.}
\setcounter{enumi}{12}
\tightlist
\item
  \textbf{minuteCaloriesWide} → Id, ActivityHour, CaloriesNN
\item
  \textbf{minuteCaloriesNarrow} → Id, ActivityHour, Calories
\item
  \textbf{hourlyCalories} → id, ActivityHour, Calories
\item
  \textbf{dailyCalories} → id, ActivityDay, Calories
\end{enumerate}

\emph{{Heart Rate}}

\begin{enumerate}
\def\labelenumi{\arabic{enumi}.}
\setcounter{enumi}{16}
\tightlist
\item
  \textbf{heartrate\_seconds} → id, time, value
\end{enumerate}

\emph{{Activity}}

\begin{enumerate}
\def\labelenumi{\arabic{enumi}.}
\setcounter{enumi}{17}
\tightlist
\item
  \textbf{dailyActivity} → id, ActivityDate, TotalSteps, TotalDistance,
  TrackerDistance, LoggedActiviesDistance, VeryActiveDistance,
  ModeratelyActiveDistance, LightActiveDistance,
  SedentaryActiveDistance, VeryActiveMinutes, FairlyActiveMinutes,
  LightlyActiveMinutes, SedentaryMinutes, Calories
\end{enumerate}

\hypertarget{code}{%
\subsection{Code}\label{code}}

For this report, I will explore how activeness impacts users' sleep
using R. To analyse that theme, I will analyze data from the following
data files:

\begin{itemize}
\tightlist
\item
  dailyActivity
\item
  dailySteps
\item
  sleepDay
\end{itemize}

\hypertarget{importing-packages}{%
\subsubsection{Importing Packages}\label{importing-packages}}

\begin{Shaded}
\begin{Highlighting}[]
\FunctionTok{install.packages}\NormalTok{(}\FunctionTok{c}\NormalTok{(}\StringTok{"tidyverse"}\NormalTok{, }\StringTok{"dplyr"}\NormalTok{, }\StringTok{"tidyr"}\NormalTok{, }\StringTok{"janitor"}\NormalTok{, }\StringTok{"readr"}\NormalTok{, }\StringTok{"skimr"}\NormalTok{, }\StringTok{"ggplot2"}\NormalTok{), }\AttributeTok{repos =} \StringTok{"http://cran.us.r{-}project.org"}\NormalTok{)}
\end{Highlighting}
\end{Shaded}

\begin{verbatim}
## Installing packages into 'C:/Users/Angela/OneDrive/Documents/R/win-library/4.1'
## (as 'lib' is unspecified)
\end{verbatim}

\begin{verbatim}
## package 'tidyverse' successfully unpacked and MD5 sums checked
## package 'dplyr' successfully unpacked and MD5 sums checked
## package 'tidyr' successfully unpacked and MD5 sums checked
## package 'janitor' successfully unpacked and MD5 sums checked
## package 'readr' successfully unpacked and MD5 sums checked
## package 'skimr' successfully unpacked and MD5 sums checked
## package 'ggplot2' successfully unpacked and MD5 sums checked
## 
## The downloaded binary packages are in
##  C:\Users\Angela\AppData\Local\Temp\Rtmpy0xTVR\downloaded_packages
\end{verbatim}

\hypertarget{loading-packages}{%
\subsubsection{Loading Packages}\label{loading-packages}}

\begin{Shaded}
\begin{Highlighting}[]
\FunctionTok{library}\NormalTok{(lubridate)}
\end{Highlighting}
\end{Shaded}

\begin{verbatim}
## Warning: package 'lubridate' was built under R version 4.1.3
\end{verbatim}

\begin{verbatim}
## 
## Attaching package: 'lubridate'
\end{verbatim}

\begin{verbatim}
## The following objects are masked from 'package:base':
## 
##     date, intersect, setdiff, union
\end{verbatim}

\begin{Shaded}
\begin{Highlighting}[]
\FunctionTok{library}\NormalTok{(dplyr)}
\end{Highlighting}
\end{Shaded}

\begin{verbatim}
## Warning: package 'dplyr' was built under R version 4.1.3
\end{verbatim}

\begin{verbatim}
## 
## Attaching package: 'dplyr'
\end{verbatim}

\begin{verbatim}
## The following objects are masked from 'package:stats':
## 
##     filter, lag
\end{verbatim}

\begin{verbatim}
## The following objects are masked from 'package:base':
## 
##     intersect, setdiff, setequal, union
\end{verbatim}

\begin{Shaded}
\begin{Highlighting}[]
\FunctionTok{library}\NormalTok{(tidyr)}
\end{Highlighting}
\end{Shaded}

\begin{verbatim}
## Warning: package 'tidyr' was built under R version 4.1.3
\end{verbatim}

\begin{Shaded}
\begin{Highlighting}[]
\FunctionTok{library}\NormalTok{(readr)}
\end{Highlighting}
\end{Shaded}

\begin{verbatim}
## Warning: package 'readr' was built under R version 4.1.3
\end{verbatim}

\begin{Shaded}
\begin{Highlighting}[]
\FunctionTok{library}\NormalTok{(janitor)}
\end{Highlighting}
\end{Shaded}

\begin{verbatim}
## Warning: package 'janitor' was built under R version 4.1.3
\end{verbatim}

\begin{verbatim}
## 
## Attaching package: 'janitor'
\end{verbatim}

\begin{verbatim}
## The following objects are masked from 'package:stats':
## 
##     chisq.test, fisher.test
\end{verbatim}

\begin{Shaded}
\begin{Highlighting}[]
\FunctionTok{library}\NormalTok{(skimr)}
\end{Highlighting}
\end{Shaded}

\begin{verbatim}
## Warning: package 'skimr' was built under R version 4.1.3
\end{verbatim}

\hypertarget{importing-data}{%
\subsubsection{Importing Data}\label{importing-data}}

\begin{verbatim}
## Rows: 940 Columns: 15
## -- Column specification --------------------------------------------------------
## Delimiter: ","
## chr  (1): ActivityDate
## dbl (14): Id, TotalSteps, TotalDistance, TrackerDistance, LoggedActivitiesDi...
## 
## i Use `spec()` to retrieve the full column specification for this data.
## i Specify the column types or set `show_col_types = FALSE` to quiet this message.
## Rows: 940 Columns: 3
## -- Column specification --------------------------------------------------------
## Delimiter: ","
## chr (1): ActivityDay
## dbl (2): Id, StepTotal
## 
## i Use `spec()` to retrieve the full column specification for this data.
## i Specify the column types or set `show_col_types = FALSE` to quiet this message.
\end{verbatim}

\hypertarget{taking-a-first-look-at-our-data}{%
\subsubsection{Taking a First Look at Our
Data}\label{taking-a-first-look-at-our-data}}

\begin{Shaded}
\begin{Highlighting}[]
\CommentTok{\#generals to our data}
\FunctionTok{head}\NormalTok{(dailyActivity)}
\end{Highlighting}
\end{Shaded}

\begin{verbatim}
## # A tibble: 6 x 15
##       Id Activ~1 Total~2 Total~3 Track~4 Logge~5 VeryA~6 Moder~7 Light~8 Seden~9
##    <dbl> <chr>     <dbl>   <dbl>   <dbl>   <dbl>   <dbl>   <dbl>   <dbl>   <dbl>
## 1 1.50e9 4/12/2~   13162    8.5     8.5        0    1.88   0.550    6.06       0
## 2 1.50e9 4/13/2~   10735    6.97    6.97       0    1.57   0.690    4.71       0
## 3 1.50e9 4/14/2~   10460    6.74    6.74       0    2.44   0.400    3.91       0
## 4 1.50e9 4/15/2~    9762    6.28    6.28       0    2.14   1.26     2.83       0
## 5 1.50e9 4/16/2~   12669    8.16    8.16       0    2.71   0.410    5.04       0
## 6 1.50e9 4/17/2~    9705    6.48    6.48       0    3.19   0.780    2.51       0
## # ... with 5 more variables: VeryActiveMinutes <dbl>,
## #   FairlyActiveMinutes <dbl>, LightlyActiveMinutes <dbl>,
## #   SedentaryMinutes <dbl>, Calories <dbl>, and abbreviated variable names
## #   1: ActivityDate, 2: TotalSteps, 3: TotalDistance, 4: TrackerDistance,
## #   5: LoggedActivitiesDistance, 6: VeryActiveDistance,
## #   7: ModeratelyActiveDistance, 8: LightActiveDistance,
## #   9: SedentaryActiveDistance
## # i Use `colnames()` to see all variable names
\end{verbatim}

\begin{Shaded}
\begin{Highlighting}[]
\FunctionTok{skim\_without\_charts}\NormalTok{(dailyActivity) }\CommentTok{\#no missing values}
\end{Highlighting}
\end{Shaded}

\begin{longtable}[]{@{}ll@{}}
\caption{Data summary}\tabularnewline
\toprule
& \\
\midrule
\endfirsthead
\toprule
& \\
\midrule
\endhead
Name & dailyActivity \\
Number of rows & 940 \\
Number of columns & 15 \\
\_\_\_\_\_\_\_\_\_\_\_\_\_\_\_\_\_\_\_\_\_\_\_ & \\
Column type frequency: & \\
character & 1 \\
numeric & 14 \\
\_\_\_\_\_\_\_\_\_\_\_\_\_\_\_\_\_\_\_\_\_\_\_\_ & \\
Group variables & None \\
\bottomrule
\end{longtable}

\textbf{Variable type: character}

\begin{longtable}[]{@{}lrrrrrrr@{}}
\toprule
skim\_variable & n\_missing & complete\_rate & min & max & empty &
n\_unique & whitespace \\
\midrule
\endhead
ActivityDate & 0 & 1 & 8 & 9 & 0 & 31 & 0 \\
\bottomrule
\end{longtable}

\textbf{Variable type: numeric}

\begin{longtable}[]{@{}
  >{\raggedright\arraybackslash}p{(\columnwidth - 18\tabcolsep) * \real{0.18}}
  >{\raggedleft\arraybackslash}p{(\columnwidth - 18\tabcolsep) * \real{0.07}}
  >{\raggedleft\arraybackslash}p{(\columnwidth - 18\tabcolsep) * \real{0.10}}
  >{\raggedleft\arraybackslash}p{(\columnwidth - 18\tabcolsep) * \real{0.09}}
  >{\raggedleft\arraybackslash}p{(\columnwidth - 18\tabcolsep) * \real{0.09}}
  >{\raggedleft\arraybackslash}p{(\columnwidth - 18\tabcolsep) * \real{0.08}}
  >{\raggedleft\arraybackslash}p{(\columnwidth - 18\tabcolsep) * \real{0.09}}
  >{\raggedleft\arraybackslash}p{(\columnwidth - 18\tabcolsep) * \real{0.09}}
  >{\raggedleft\arraybackslash}p{(\columnwidth - 18\tabcolsep) * \real{0.09}}
  >{\raggedleft\arraybackslash}p{(\columnwidth - 18\tabcolsep) * \real{0.09}}@{}}
\toprule
skim\_variable & n\_missing & complete\_rate & mean & sd & p0 & p25 &
p50 & p75 & p100 \\
\midrule
\endhead
Id & 0 & 1 & 4.855407e+09 & 2.424805e+09 & 1503960366 & 2.320127e+09 &
4.445115e+09 & 6.962181e+09 & 8.877689e+09 \\
TotalSteps & 0 & 1 & 7.637910e+03 & 5.087150e+03 & 0 & 3.789750e+03 &
7.405500e+03 & 1.072700e+04 & 3.601900e+04 \\
TotalDistance & 0 & 1 & 5.490000e+00 & 3.920000e+00 & 0 & 2.620000e+00 &
5.240000e+00 & 7.710000e+00 & 2.803000e+01 \\
TrackerDistance & 0 & 1 & 5.480000e+00 & 3.910000e+00 & 0 & 2.620000e+00
& 5.240000e+00 & 7.710000e+00 & 2.803000e+01 \\
LoggedActivitiesDistance & 0 & 1 & 1.100000e-01 & 6.200000e-01 & 0 &
0.000000e+00 & 0.000000e+00 & 0.000000e+00 & 4.940000e+00 \\
VeryActiveDistance & 0 & 1 & 1.500000e+00 & 2.660000e+00 & 0 &
0.000000e+00 & 2.100000e-01 & 2.050000e+00 & 2.192000e+01 \\
ModeratelyActiveDistance & 0 & 1 & 5.700000e-01 & 8.800000e-01 & 0 &
0.000000e+00 & 2.400000e-01 & 8.000000e-01 & 6.480000e+00 \\
LightActiveDistance & 0 & 1 & 3.340000e+00 & 2.040000e+00 & 0 &
1.950000e+00 & 3.360000e+00 & 4.780000e+00 & 1.071000e+01 \\
SedentaryActiveDistance & 0 & 1 & 0.000000e+00 & 1.000000e-02 & 0 &
0.000000e+00 & 0.000000e+00 & 0.000000e+00 & 1.100000e-01 \\
VeryActiveMinutes & 0 & 1 & 2.116000e+01 & 3.284000e+01 & 0 &
0.000000e+00 & 4.000000e+00 & 3.200000e+01 & 2.100000e+02 \\
FairlyActiveMinutes & 0 & 1 & 1.356000e+01 & 1.999000e+01 & 0 &
0.000000e+00 & 6.000000e+00 & 1.900000e+01 & 1.430000e+02 \\
LightlyActiveMinutes & 0 & 1 & 1.928100e+02 & 1.091700e+02 & 0 &
1.270000e+02 & 1.990000e+02 & 2.640000e+02 & 5.180000e+02 \\
SedentaryMinutes & 0 & 1 & 9.912100e+02 & 3.012700e+02 & 0 &
7.297500e+02 & 1.057500e+03 & 1.229500e+03 & 1.440000e+03 \\
Calories & 0 & 1 & 2.303610e+03 & 7.181700e+02 & 0 & 1.828500e+03 &
2.134000e+03 & 2.793250e+03 & 4.900000e+03 \\
\bottomrule
\end{longtable}

\begin{Shaded}
\begin{Highlighting}[]
\FunctionTok{str}\NormalTok{(dailyActivity)}
\end{Highlighting}
\end{Shaded}

\begin{verbatim}
## spec_tbl_df [940 x 15] (S3: spec_tbl_df/tbl_df/tbl/data.frame)
##  $ Id                      : num [1:940] 1.5e+09 1.5e+09 1.5e+09 1.5e+09 1.5e+09 ...
##  $ ActivityDate            : chr [1:940] "4/12/2016" "4/13/2016" "4/14/2016" "4/15/2016" ...
##  $ TotalSteps              : num [1:940] 13162 10735 10460 9762 12669 ...
##  $ TotalDistance           : num [1:940] 8.5 6.97 6.74 6.28 8.16 ...
##  $ TrackerDistance         : num [1:940] 8.5 6.97 6.74 6.28 8.16 ...
##  $ LoggedActivitiesDistance: num [1:940] 0 0 0 0 0 0 0 0 0 0 ...
##  $ VeryActiveDistance      : num [1:940] 1.88 1.57 2.44 2.14 2.71 ...
##  $ ModeratelyActiveDistance: num [1:940] 0.55 0.69 0.4 1.26 0.41 ...
##  $ LightActiveDistance     : num [1:940] 6.06 4.71 3.91 2.83 5.04 ...
##  $ SedentaryActiveDistance : num [1:940] 0 0 0 0 0 0 0 0 0 0 ...
##  $ VeryActiveMinutes       : num [1:940] 25 21 30 29 36 38 42 50 28 19 ...
##  $ FairlyActiveMinutes     : num [1:940] 13 19 11 34 10 20 16 31 12 8 ...
##  $ LightlyActiveMinutes    : num [1:940] 328 217 181 209 221 164 233 264 205 211 ...
##  $ SedentaryMinutes        : num [1:940] 728 776 1218 726 773 ...
##  $ Calories                : num [1:940] 1985 1797 1776 1745 1863 ...
##  - attr(*, "spec")=
##   .. cols(
##   ..   Id = col_double(),
##   ..   ActivityDate = col_character(),
##   ..   TotalSteps = col_double(),
##   ..   TotalDistance = col_double(),
##   ..   TrackerDistance = col_double(),
##   ..   LoggedActivitiesDistance = col_double(),
##   ..   VeryActiveDistance = col_double(),
##   ..   ModeratelyActiveDistance = col_double(),
##   ..   LightActiveDistance = col_double(),
##   ..   SedentaryActiveDistance = col_double(),
##   ..   VeryActiveMinutes = col_double(),
##   ..   FairlyActiveMinutes = col_double(),
##   ..   LightlyActiveMinutes = col_double(),
##   ..   SedentaryMinutes = col_double(),
##   ..   Calories = col_double()
##   .. )
##  - attr(*, "problems")=<externalptr>
\end{verbatim}

\begin{Shaded}
\begin{Highlighting}[]
\FunctionTok{head}\NormalTok{(dailySteps)}
\end{Highlighting}
\end{Shaded}

\begin{verbatim}
## # A tibble: 6 x 3
##           Id ActivityDay StepTotal
##        <dbl> <chr>           <dbl>
## 1 1503960366 4/12/2016       13162
## 2 1503960366 4/13/2016       10735
## 3 1503960366 4/14/2016       10460
## 4 1503960366 4/15/2016        9762
## 5 1503960366 4/16/2016       12669
## 6 1503960366 4/17/2016        9705
\end{verbatim}

\begin{Shaded}
\begin{Highlighting}[]
\FunctionTok{skim\_without\_charts}\NormalTok{(dailySteps) }\CommentTok{\#no missing values}
\end{Highlighting}
\end{Shaded}

\begin{longtable}[]{@{}ll@{}}
\caption{Data summary}\tabularnewline
\toprule
& \\
\midrule
\endfirsthead
\toprule
& \\
\midrule
\endhead
Name & dailySteps \\
Number of rows & 940 \\
Number of columns & 3 \\
\_\_\_\_\_\_\_\_\_\_\_\_\_\_\_\_\_\_\_\_\_\_\_ & \\
Column type frequency: & \\
character & 1 \\
numeric & 2 \\
\_\_\_\_\_\_\_\_\_\_\_\_\_\_\_\_\_\_\_\_\_\_\_\_ & \\
Group variables & None \\
\bottomrule
\end{longtable}

\textbf{Variable type: character}

\begin{longtable}[]{@{}lrrrrrrr@{}}
\toprule
skim\_variable & n\_missing & complete\_rate & min & max & empty &
n\_unique & whitespace \\
\midrule
\endhead
ActivityDay & 0 & 1 & 8 & 9 & 0 & 31 & 0 \\
\bottomrule
\end{longtable}

\textbf{Variable type: numeric}

\begin{longtable}[]{@{}
  >{\raggedright\arraybackslash}p{(\columnwidth - 18\tabcolsep) * \real{0.11}}
  >{\raggedleft\arraybackslash}p{(\columnwidth - 18\tabcolsep) * \real{0.08}}
  >{\raggedleft\arraybackslash}p{(\columnwidth - 18\tabcolsep) * \real{0.11}}
  >{\raggedleft\arraybackslash}p{(\columnwidth - 18\tabcolsep) * \real{0.11}}
  >{\raggedleft\arraybackslash}p{(\columnwidth - 18\tabcolsep) * \real{0.11}}
  >{\raggedleft\arraybackslash}p{(\columnwidth - 18\tabcolsep) * \real{0.09}}
  >{\raggedleft\arraybackslash}p{(\columnwidth - 18\tabcolsep) * \real{0.11}}
  >{\raggedleft\arraybackslash}p{(\columnwidth - 18\tabcolsep) * \real{0.11}}
  >{\raggedleft\arraybackslash}p{(\columnwidth - 18\tabcolsep) * \real{0.09}}
  >{\raggedleft\arraybackslash}p{(\columnwidth - 18\tabcolsep) * \real{0.09}}@{}}
\toprule
skim\_variable & n\_missing & complete\_rate & mean & sd & p0 & p25 &
p50 & p75 & p100 \\
\midrule
\endhead
Id & 0 & 1 & 4.855407e+09 & 2.424805e+09 & 1503960366 & 2.320127e+09 &
4445114986.0 & 6962181067 & 8877689391 \\
StepTotal & 0 & 1 & 7.637910e+03 & 5.087150e+03 & 0 & 3.789750e+03 &
7405.5 & 10727 & 36019 \\
\bottomrule
\end{longtable}

\begin{Shaded}
\begin{Highlighting}[]
\FunctionTok{str}\NormalTok{(dailySteps)}
\end{Highlighting}
\end{Shaded}

\begin{verbatim}
## spec_tbl_df [940 x 3] (S3: spec_tbl_df/tbl_df/tbl/data.frame)
##  $ Id         : num [1:940] 1.5e+09 1.5e+09 1.5e+09 1.5e+09 1.5e+09 ...
##  $ ActivityDay: chr [1:940] "4/12/2016" "4/13/2016" "4/14/2016" "4/15/2016" ...
##  $ StepTotal  : num [1:940] 13162 10735 10460 9762 12669 ...
##  - attr(*, "spec")=
##   .. cols(
##   ..   Id = col_double(),
##   ..   ActivityDay = col_character(),
##   ..   StepTotal = col_double()
##   .. )
##  - attr(*, "problems")=<externalptr>
\end{verbatim}

\begin{Shaded}
\begin{Highlighting}[]
\FunctionTok{head}\NormalTok{(sleepDay)}
\end{Highlighting}
\end{Shaded}

\begin{verbatim}
##           Id              SleepDay TotalSleepRecords TotalMinutesAsleep
## 1 1503960366 4/12/2016 12:00:00 AM                 1                327
## 2 1503960366 4/13/2016 12:00:00 AM                 2                384
## 3 1503960366 4/15/2016 12:00:00 AM                 1                412
## 4 1503960366 4/16/2016 12:00:00 AM                 2                340
## 5 1503960366 4/17/2016 12:00:00 AM                 1                700
## 6 1503960366 4/19/2016 12:00:00 AM                 1                304
##   TotalTimeInBed
## 1            346
## 2            407
## 3            442
## 4            367
## 5            712
## 6            320
\end{verbatim}

\begin{Shaded}
\begin{Highlighting}[]
\FunctionTok{skim\_without\_charts}\NormalTok{(sleepDay) }\CommentTok{\#no missing values}
\end{Highlighting}
\end{Shaded}

\begin{longtable}[]{@{}ll@{}}
\caption{Data summary}\tabularnewline
\toprule
& \\
\midrule
\endfirsthead
\toprule
& \\
\midrule
\endhead
Name & sleepDay \\
Number of rows & 413 \\
Number of columns & 5 \\
\_\_\_\_\_\_\_\_\_\_\_\_\_\_\_\_\_\_\_\_\_\_\_ & \\
Column type frequency: & \\
character & 1 \\
numeric & 4 \\
\_\_\_\_\_\_\_\_\_\_\_\_\_\_\_\_\_\_\_\_\_\_\_\_ & \\
Group variables & None \\
\bottomrule
\end{longtable}

\textbf{Variable type: character}

\begin{longtable}[]{@{}lrrrrrrr@{}}
\toprule
skim\_variable & n\_missing & complete\_rate & min & max & empty &
n\_unique & whitespace \\
\midrule
\endhead
SleepDay & 0 & 1 & 20 & 21 & 0 & 31 & 0 \\
\bottomrule
\end{longtable}

\textbf{Variable type: numeric}

\begin{longtable}[]{@{}
  >{\raggedright\arraybackslash}p{(\columnwidth - 18\tabcolsep) * \real{0.15}}
  >{\raggedleft\arraybackslash}p{(\columnwidth - 18\tabcolsep) * \real{0.08}}
  >{\raggedleft\arraybackslash}p{(\columnwidth - 18\tabcolsep) * \real{0.11}}
  >{\raggedleft\arraybackslash}p{(\columnwidth - 18\tabcolsep) * \real{0.11}}
  >{\raggedleft\arraybackslash}p{(\columnwidth - 18\tabcolsep) * \real{0.10}}
  >{\raggedleft\arraybackslash}p{(\columnwidth - 18\tabcolsep) * \real{0.09}}
  >{\raggedleft\arraybackslash}p{(\columnwidth - 18\tabcolsep) * \real{0.09}}
  >{\raggedleft\arraybackslash}p{(\columnwidth - 18\tabcolsep) * \real{0.09}}
  >{\raggedleft\arraybackslash}p{(\columnwidth - 18\tabcolsep) * \real{0.09}}
  >{\raggedleft\arraybackslash}p{(\columnwidth - 18\tabcolsep) * \real{0.09}}@{}}
\toprule
skim\_variable & n\_missing & complete\_rate & mean & sd & p0 & p25 &
p50 & p75 & p100 \\
\midrule
\endhead
Id & 0 & 1 & 5.000979e+09 & 2.06036e+09 & 1503960366 & 3977333714 &
4702921684 & 6962181067 & 8792009665 \\
TotalSleepRecords & 0 & 1 & 1.120000e+00 & 3.50000e-01 & 1 & 1 & 1 & 1 &
3 \\
TotalMinutesAsleep & 0 & 1 & 4.194700e+02 & 1.18340e+02 & 58 & 361 & 433
& 490 & 796 \\
TotalTimeInBed & 0 & 1 & 4.586400e+02 & 1.27100e+02 & 61 & 403 & 463 &
526 & 961 \\
\bottomrule
\end{longtable}

\begin{Shaded}
\begin{Highlighting}[]
\FunctionTok{str}\NormalTok{(sleepDay)}
\end{Highlighting}
\end{Shaded}

\begin{verbatim}
## 'data.frame':    413 obs. of  5 variables:
##  $ Id                : num  1.5e+09 1.5e+09 1.5e+09 1.5e+09 1.5e+09 ...
##  $ SleepDay          : chr  "4/12/2016 12:00:00 AM" "4/13/2016 12:00:00 AM" "4/15/2016 12:00:00 AM" "4/16/2016 12:00:00 AM" ...
##  $ TotalSleepRecords : int  1 2 1 2 1 1 1 1 1 1 ...
##  $ TotalMinutesAsleep: int  327 384 412 340 700 304 360 325 361 430 ...
##  $ TotalTimeInBed    : int  346 407 442 367 712 320 377 364 384 449 ...
\end{verbatim}

\hypertarget{cleaning-the-dataset}{%
\subsubsection{Cleaning the Dataset}\label{cleaning-the-dataset}}

\begin{Shaded}
\begin{Highlighting}[]
\CommentTok{\#Cleaning{-} formatting dates from string to date and correcting data types}
\CommentTok{\#Cleaning{-} adding another column \textquotesingle{}Date\textquotesingle{} for activity and sleep so that I can merge later}
\NormalTok{activity }\OtherTok{\textless{}{-}}\NormalTok{ dailyActivity }\SpecialCharTok{\%\textgreater{}\%} 
  \FunctionTok{mutate}\NormalTok{(}\AttributeTok{ActivityDate =} \FunctionTok{mdy}\NormalTok{(ActivityDate)) }\SpecialCharTok{\%\textgreater{}\%} 
  \FunctionTok{mutate}\NormalTok{(}\AttributeTok{TotalSteps =} \FunctionTok{as.integer}\NormalTok{(TotalSteps)) }\SpecialCharTok{\%\textgreater{}\%} 
  \FunctionTok{mutate}\NormalTok{(}\AttributeTok{Date =}\NormalTok{ ActivityDate)}

\NormalTok{sleep }\OtherTok{\textless{}{-}}\NormalTok{ sleepDay }\SpecialCharTok{\%\textgreater{}\%} 
  \FunctionTok{mutate}\NormalTok{(}\AttributeTok{SleepDay =} \FunctionTok{mdy\_hms}\NormalTok{(SleepDay)) }\SpecialCharTok{\%\textgreater{}\%} 
  \FunctionTok{mutate}\NormalTok{(}\AttributeTok{TotalMinutesAsleep =} \FunctionTok{as.numeric}\NormalTok{(TotalMinutesAsleep)) }\SpecialCharTok{\%\textgreater{}\%} 
  \FunctionTok{mutate}\NormalTok{(}\AttributeTok{TotalTimeInBed =} \FunctionTok{as.numeric}\NormalTok{(TotalTimeInBed)) }\SpecialCharTok{\%\textgreater{}\%} 
  \FunctionTok{mutate}\NormalTok{(}\AttributeTok{Date =}\NormalTok{ SleepDay)}

\NormalTok{steps }\OtherTok{\textless{}{-}}\NormalTok{ dailySteps }\SpecialCharTok{\%\textgreater{}\%} 
  \FunctionTok{mutate}\NormalTok{(}\AttributeTok{ActivityDay =} \FunctionTok{mdy}\NormalTok{(ActivityDay)) }\SpecialCharTok{\%\textgreater{}\%} 
  \FunctionTok{mutate}\NormalTok{(}\AttributeTok{Date =}\NormalTok{ ActivityDay)}
\end{Highlighting}
\end{Shaded}

\hypertarget{exploring-data}{%
\subsubsection{Exploring Data}\label{exploring-data}}

\begin{Shaded}
\begin{Highlighting}[]
\NormalTok{activity }\SpecialCharTok{\%\textgreater{}\%} 
  \FunctionTok{select}\NormalTok{(TotalSteps, VeryActiveMinutes, FairlyActiveMinutes, }
\NormalTok{         LightlyActiveMinutes, SedentaryMinutes) }\SpecialCharTok{\%\textgreater{}\%} 
  \FunctionTok{summary}\NormalTok{(activity)}
\end{Highlighting}
\end{Shaded}

\begin{verbatim}
##    TotalSteps    VeryActiveMinutes FairlyActiveMinutes LightlyActiveMinutes
##  Min.   :    0   Min.   :  0.00    Min.   :  0.00      Min.   :  0.0       
##  1st Qu.: 3790   1st Qu.:  0.00    1st Qu.:  0.00      1st Qu.:127.0       
##  Median : 7406   Median :  4.00    Median :  6.00      Median :199.0       
##  Mean   : 7638   Mean   : 21.16    Mean   : 13.56      Mean   :192.8       
##  3rd Qu.:10727   3rd Qu.: 32.00    3rd Qu.: 19.00      3rd Qu.:264.0       
##  Max.   :36019   Max.   :210.00    Max.   :143.00      Max.   :518.0       
##  SedentaryMinutes
##  Min.   :   0.0  
##  1st Qu.: 729.8  
##  Median :1057.5  
##  Mean   : 991.2  
##  3rd Qu.:1229.5  
##  Max.   :1440.0
\end{verbatim}

\begin{Shaded}
\begin{Highlighting}[]
\NormalTok{sleep }\SpecialCharTok{\%\textgreater{}\%} 
  \FunctionTok{select}\NormalTok{(Date, TotalSleepRecords, TotalMinutesAsleep, TotalTimeInBed) }\SpecialCharTok{\%\textgreater{}\%} 
  \FunctionTok{summary}\NormalTok{()}
\end{Highlighting}
\end{Shaded}

\begin{verbatim}
##       Date                     TotalSleepRecords TotalMinutesAsleep
##  Min.   :2016-04-12 00:00:00   Min.   :1.000     Min.   : 58.0     
##  1st Qu.:2016-04-19 00:00:00   1st Qu.:1.000     1st Qu.:361.0     
##  Median :2016-04-27 00:00:00   Median :1.000     Median :433.0     
##  Mean   :2016-04-26 12:40:05   Mean   :1.119     Mean   :419.5     
##  3rd Qu.:2016-05-04 00:00:00   3rd Qu.:1.000     3rd Qu.:490.0     
##  Max.   :2016-05-12 00:00:00   Max.   :3.000     Max.   :796.0     
##  TotalTimeInBed 
##  Min.   : 61.0  
##  1st Qu.:403.0  
##  Median :463.0  
##  Mean   :458.6  
##  3rd Qu.:526.0  
##  Max.   :961.0
\end{verbatim}

\begin{Shaded}
\begin{Highlighting}[]
\NormalTok{steps }\SpecialCharTok{\%\textgreater{}\%} 
  \FunctionTok{select}\NormalTok{(StepTotal) }\SpecialCharTok{\%\textgreater{}\%} 
  \FunctionTok{summary}\NormalTok{()}
\end{Highlighting}
\end{Shaded}

\begin{verbatim}
##    StepTotal    
##  Min.   :    0  
##  1st Qu.: 3790  
##  Median : 7406  
##  Mean   : 7638  
##  3rd Qu.:10727  
##  Max.   :36019
\end{verbatim}

\begin{Shaded}
\begin{Highlighting}[]
\CommentTok{\#Exploring{-} distinct users in each data file}
\NormalTok{users\_activity }\OtherTok{\textless{}{-}} \FunctionTok{unique}\NormalTok{(dailyActivity}\SpecialCharTok{$}\NormalTok{Id)}
\FunctionTok{length}\NormalTok{(users\_activity) }\CommentTok{\#33}
\end{Highlighting}
\end{Shaded}

\begin{verbatim}
## [1] 33
\end{verbatim}

\begin{Shaded}
\begin{Highlighting}[]
\NormalTok{users\_steps }\OtherTok{\textless{}{-}} \FunctionTok{unique}\NormalTok{(dailySteps}\SpecialCharTok{$}\NormalTok{Id)}
\FunctionTok{length}\NormalTok{(users\_steps) }\CommentTok{\#33}
\end{Highlighting}
\end{Shaded}

\begin{verbatim}
## [1] 33
\end{verbatim}

\begin{Shaded}
\begin{Highlighting}[]
\NormalTok{users\_sleep }\OtherTok{\textless{}{-}} \FunctionTok{unique}\NormalTok{(sleepDay}\SpecialCharTok{$}\NormalTok{Id)}
\FunctionTok{length}\NormalTok{(users\_sleep) }\CommentTok{\#24}
\end{Highlighting}
\end{Shaded}

\begin{verbatim}
## [1] 24
\end{verbatim}

\begin{Shaded}
\begin{Highlighting}[]
\CommentTok{\#Exploring{-} counting number of entries}
\FunctionTok{nrow}\NormalTok{(sleep) }\CommentTok{\#413}
\end{Highlighting}
\end{Shaded}

\begin{verbatim}
## [1] 413
\end{verbatim}

\begin{Shaded}
\begin{Highlighting}[]
\FunctionTok{nrow}\NormalTok{(activity) }\CommentTok{\#940}
\end{Highlighting}
\end{Shaded}

\begin{verbatim}
## [1] 940
\end{verbatim}

\begin{Shaded}
\begin{Highlighting}[]
\CommentTok{\#Exploring{-} how many times each user logged in}
\NormalTok{sleepDay }\SpecialCharTok{\%\textgreater{}\%} \FunctionTok{count}\NormalTok{(Id)}
\end{Highlighting}
\end{Shaded}

\begin{verbatim}
##            Id  n
## 1  1503960366 25
## 2  1644430081  4
## 3  1844505072  3
## 4  1927972279  5
## 5  2026352035 28
## 6  2320127002  1
## 7  2347167796 15
## 8  3977333714 28
## 9  4020332650  8
## 10 4319703577 26
## 11 4388161847 24
## 12 4445114986 28
## 13 4558609924  5
## 14 4702921684 28
## 15 5553957443 31
## 16 5577150313 26
## 17 6117666160 18
## 18 6775888955  3
## 19 6962181067 31
## 20 7007744171  2
## 21 7086361926 24
## 22 8053475328  3
## 23 8378563200 32
## 24 8792009665 15
\end{verbatim}

\hypertarget{analyze}{%
\section{Analyze}\label{analyze}}

Looking more deeply into these variables:

\begin{itemize}
\tightlist
\item
  TotalSteps (steps users take per day)
\item
  SedentaryMinutes (minutes users stay sedentary per day (i.e.~sitting
  and lying down))
\item
  TotalMinutesSleep (minutes users are sleeping per day)
\item
  TotalTimeInBed (minutes users are in bed)
\item
  Activity Level (very active, fairly active, lightly active, sedentary)

  \begin{itemize}
  \tightlist
  \item
    StepsActivityLevel (categorize users' activity levels according to
    recommended steps per day
    \href{https://www.medicinenet.com/how_many_steps_a_day_is_considered_active/article.htm}{1})
  \item
    ActivityLevelMins (categorize users by activity levels according to
    users daily activity minutes)
  \end{itemize}
\item
  SleepSufficiency (categorize whether users had enough sleep (at least
  420 minutes or 7 hours
  \href{https://www.sleepfoundation.org/sleep-hygiene/what-is-healthy-sleep}{2})
  or not enough sleep (less than 420 minutes or 7 hours)
\item
  AwakeInBed (minutes that users are in bed but not asleep)
\end{itemize}

\hypertarget{code-1}{%
\subsection{Code}\label{code-1}}

\hypertarget{merging-data}{%
\subsubsection{Merging Data}\label{merging-data}}

\begin{Shaded}
\begin{Highlighting}[]
\CommentTok{\#Merging activity and sleep in a df by \textquotesingle{}Id\textquotesingle{} and \textquotesingle{}Date\textquotesingle{}, excluding columns I don\textquotesingle{}t need}
\NormalTok{sleep\_activity }\OtherTok{\textless{}{-}} \FunctionTok{merge}\NormalTok{(sleep, activity, }\AttributeTok{by =} \FunctionTok{c}\NormalTok{(}\StringTok{\textquotesingle{}Id\textquotesingle{}}\NormalTok{, }\StringTok{\textquotesingle{}Date\textquotesingle{}}\NormalTok{)) }\SpecialCharTok{\%\textgreater{}\%} 
  \FunctionTok{select}\NormalTok{(}\SpecialCharTok{{-}}\FunctionTok{c}\NormalTok{(TotalDistance, TrackerDistance, LoggedActivitiesDistance, }
\NormalTok{            VeryActiveDistance, ModeratelyActiveDistance, LightActiveDistance, SedentaryActiveDistance))}
\FunctionTok{View}\NormalTok{(sleep\_activity)}

\CommentTok{\#filtering out the data that captured sedentary minutes of 1440 mins (24 hrs)}
\NormalTok{sleep\_activity\_cleaned }\OtherTok{\textless{}{-}}\NormalTok{ sleep\_activity }\SpecialCharTok{\%\textgreater{}\%} 
  \FunctionTok{filter}\NormalTok{(SedentaryMinutes }\SpecialCharTok{!=} \DecValTok{1440}\NormalTok{)}

\CommentTok{\#filtered out the data that was tracked when user was not using it}
\NormalTok{sleep\_activity\_cleaned }\OtherTok{\textless{}{-}}\NormalTok{ sleep\_activity }\SpecialCharTok{\%\textgreater{}\%} 
  \FunctionTok{filter}\NormalTok{(SedentaryMinutes }\SpecialCharTok{!=} \DecValTok{1440}\NormalTok{)}
\FunctionTok{summary}\NormalTok{(sleep\_activity\_cleaned)}
\end{Highlighting}
\end{Shaded}

\begin{verbatim}
##        Id                 Date                    
##  Min.   :1.504e+09   Min.   :2016-04-12 00:00:00  
##  1st Qu.:3.977e+09   1st Qu.:2016-04-19 00:00:00  
##  Median :4.703e+09   Median :2016-04-27 00:00:00  
##  Mean   :5.001e+09   Mean   :2016-04-26 12:40:05  
##  3rd Qu.:6.962e+09   3rd Qu.:2016-05-04 00:00:00  
##  Max.   :8.792e+09   Max.   :2016-05-12 00:00:00  
##     SleepDay                   TotalSleepRecords TotalMinutesAsleep
##  Min.   :2016-04-12 00:00:00   Min.   :1.000     Min.   : 58.0     
##  1st Qu.:2016-04-19 00:00:00   1st Qu.:1.000     1st Qu.:361.0     
##  Median :2016-04-27 00:00:00   Median :1.000     Median :433.0     
##  Mean   :2016-04-26 12:40:05   Mean   :1.119     Mean   :419.5     
##  3rd Qu.:2016-05-04 00:00:00   3rd Qu.:1.000     3rd Qu.:490.0     
##  Max.   :2016-05-12 00:00:00   Max.   :3.000     Max.   :796.0     
##  TotalTimeInBed   ActivityDate          TotalSteps    VeryActiveMinutes
##  Min.   : 61.0   Min.   :2016-04-12   Min.   :   17   Min.   :  0.00   
##  1st Qu.:403.0   1st Qu.:2016-04-19   1st Qu.: 5206   1st Qu.:  0.00   
##  Median :463.0   Median :2016-04-27   Median : 8925   Median :  9.00   
##  Mean   :458.6   Mean   :2016-04-26   Mean   : 8541   Mean   : 25.19   
##  3rd Qu.:526.0   3rd Qu.:2016-05-04   3rd Qu.:11393   3rd Qu.: 38.00   
##  Max.   :961.0   Max.   :2016-05-12   Max.   :22770   Max.   :210.00   
##  FairlyActiveMinutes LightlyActiveMinutes SedentaryMinutes    Calories   
##  Min.   :  0.00      Min.   :  2.0        Min.   :   0.0   Min.   : 257  
##  1st Qu.:  0.00      1st Qu.:158.0        1st Qu.: 631.0   1st Qu.:1850  
##  Median : 11.00      Median :208.0        Median : 717.0   Median :2220  
##  Mean   : 18.04      Mean   :216.9        Mean   : 712.2   Mean   :2398  
##  3rd Qu.: 27.00      3rd Qu.:263.0        3rd Qu.: 783.0   3rd Qu.:2926  
##  Max.   :143.00      Max.   :518.0        Max.   :1265.0   Max.   :4900
\end{verbatim}

The mean for sedentary minutes and total steps are below average for the
recommended level for healthy adults. For sedentary minutes, the average
time users have been sedentary is 717 minutes per day, almost 12 hours
which is considered a high average. The mean for total steps is 8541
which is considered low for the recommended at least 10,000 steps per
day.
\href{https://ijbnpa.biomedcentral.com/articles/10.1186/1479-5868-8-79\#:~:text=In\%20summary\%2C\%20at\%20least\%20in,populace\%20\%5B3\%2C\%2023\%5D.}{3}

\hypertarget{analyzing-data}{%
\subsubsection{Analyzing Data}\label{analyzing-data}}

On average, how many users are active? (at least 10k steps a day)

\begin{Shaded}
\begin{Highlighting}[]
\NormalTok{avg\_steps }\OtherTok{\textless{}{-}}\NormalTok{ sleep\_activity\_cleaned }\SpecialCharTok{\%\textgreater{}\%} 
  \FunctionTok{group\_by}\NormalTok{(Id) }\SpecialCharTok{\%\textgreater{}\%} 
  \FunctionTok{summarise}\NormalTok{(}\AttributeTok{AverageSteps =} \FunctionTok{mean}\NormalTok{(TotalSteps, }\AttributeTok{na.rm =} \ConstantTok{TRUE}\NormalTok{))}

\NormalTok{active\_users\_steps }\OtherTok{\textless{}{-}} \FunctionTok{filter}\NormalTok{(avg\_steps, AverageSteps }\SpecialCharTok{\textgreater{}=} \DecValTok{10000}\NormalTok{)}
\FunctionTok{count}\NormalTok{(active\_users\_steps) }\CommentTok{\#5}
\end{Highlighting}
\end{Shaded}

\begin{verbatim}
## # A tibble: 1 x 1
##       n
##   <int>
## 1     5
\end{verbatim}

\begin{Shaded}
\begin{Highlighting}[]
\NormalTok{active\_users\_steps\_percent }\OtherTok{\textless{}{-}} \FunctionTok{nrow}\NormalTok{(active\_users\_steps)}\SpecialCharTok{/}\FunctionTok{nrow}\NormalTok{(avg\_steps)}\SpecialCharTok{*}\DecValTok{100}
\FunctionTok{print}\NormalTok{(active\_users\_steps\_percent ) }\CommentTok{\#20.83\%}
\end{Highlighting}
\end{Shaded}

\begin{verbatim}
## [1] 20.83333
\end{verbatim}

5 out of 24 or around 21\% of smart device users are considered active
or walk at least 10,000 steps a day. That is a relatively low number,
considering that smart device trackers are meant to measure users'
activity and health to motivate users towards their fitness goals.

How many users are sedentary? (less than 5k steps a day)

\begin{Shaded}
\begin{Highlighting}[]
\NormalTok{sedentary\_users\_steps }\OtherTok{\textless{}{-}} \FunctionTok{filter}\NormalTok{(avg\_steps, AverageSteps }\SpecialCharTok{\textless{}} \DecValTok{5000}\NormalTok{)}
\FunctionTok{count}\NormalTok{(sedentary\_users\_steps) }\CommentTok{\#5}
\end{Highlighting}
\end{Shaded}

\begin{verbatim}
## # A tibble: 1 x 1
##       n
##   <int>
## 1     5
\end{verbatim}

\begin{Shaded}
\begin{Highlighting}[]
\NormalTok{sedentary\_users\_steps\_percent }\OtherTok{\textless{}{-}} \FunctionTok{nrow}\NormalTok{(sedentary\_users\_steps)}\SpecialCharTok{/}\FunctionTok{nrow}\NormalTok{(avg\_steps)}\SpecialCharTok{*}\DecValTok{100}
\FunctionTok{print}\NormalTok{(sedentary\_users\_steps\_percent) }\CommentTok{\#20.83\%}
\end{Highlighting}
\end{Shaded}

\begin{verbatim}
## [1] 20.83333
\end{verbatim}

5 out of 33 or around 21\% of users are considered sedentary or
inactive, meaning they walk less than 5,000 steps per day. That is a
relatively low number which is good because we want to encourage more
exercise and activeness.

On average, how many users are healthy sleepers? (at least 420 minutes a
night)

\begin{Shaded}
\begin{Highlighting}[]
\NormalTok{avg\_sleep\_minutes }\OtherTok{\textless{}{-}}\NormalTok{ sleep\_activity\_cleaned }\SpecialCharTok{\%\textgreater{}\%} 
  \FunctionTok{group\_by}\NormalTok{(Id) }\SpecialCharTok{\%\textgreater{}\%} 
  \FunctionTok{summarise}\NormalTok{(}\AttributeTok{AverageSleepMinutes =} \FunctionTok{mean}\NormalTok{(TotalMinutesAsleep, }\AttributeTok{na.rm =} \ConstantTok{TRUE}\NormalTok{))}
\FunctionTok{summary}\NormalTok{(avg\_sleep\_minutes}\SpecialCharTok{$}\NormalTok{AverageSleepMinutes) }\CommentTok{\#377.6 or 6.29 hrs}
\end{Highlighting}
\end{Shaded}

\begin{verbatim}
##    Min. 1st Qu.  Median    Mean 3rd Qu.    Max. 
##    61.0   336.3   419.1   377.6   449.3   652.0
\end{verbatim}

On average, users sleep for 377.6 minutes or 6.29 hours. That average
falls below the recommended level of 420 minutes or 7 hours
\href{https://www.sleepfoundation.org/sleep-hygiene/what-is-healthy-sleep}{2}.

\begin{Shaded}
\begin{Highlighting}[]
\CommentTok{\#create new columns in sleep\_activity\_cleaned df}
\NormalTok{sleep\_activity\_cleaned}\SpecialCharTok{$}\NormalTok{TotalLoggedMinutes }\OtherTok{=}\NormalTok{ sleep\_activity\_cleaned}\SpecialCharTok{$}\NormalTok{TotalTimeInBed }\SpecialCharTok{+} 
\NormalTok{  sleep\_activity\_cleaned}\SpecialCharTok{$}\NormalTok{VeryActiveMinutes }\SpecialCharTok{+} 
\NormalTok{  sleep\_activity\_cleaned}\SpecialCharTok{$}\NormalTok{FairlyActiveMinutes }\SpecialCharTok{+} 
\NormalTok{  sleep\_activity\_cleaned}\SpecialCharTok{$}\NormalTok{LightlyActiveMinutes }\SpecialCharTok{+} 
\NormalTok{  sleep\_activity\_cleaned}\SpecialCharTok{$}\NormalTok{SedentaryMinutes}

\NormalTok{sleep\_activity\_cleaned}\SpecialCharTok{$}\NormalTok{TotalActiveMinutes }\OtherTok{=}\NormalTok{ sleep\_activity\_cleaned}\SpecialCharTok{$}\NormalTok{VeryActiveMinutes }\SpecialCharTok{+} 
\NormalTok{  sleep\_activity\_cleaned}\SpecialCharTok{$}\NormalTok{FairlyActiveMinutes }\SpecialCharTok{+} 
\NormalTok{  sleep\_activity\_cleaned}\SpecialCharTok{$}\NormalTok{LightlyActiveMinutes}


\CommentTok{\#creating new column to categorize the activity level of users based on steps and determine whether they had sufficient sleep}
\NormalTok{sleep\_steps }\OtherTok{\textless{}{-}}\NormalTok{ sleep\_activity\_cleaned }\SpecialCharTok{\%\textgreater{}\%} 
  \FunctionTok{mutate}\NormalTok{(}\AttributeTok{StepsActivityLevel =} \FunctionTok{case\_when}\NormalTok{(}
\NormalTok{    TotalSteps }\SpecialCharTok{\textless{}} \DecValTok{5000} \SpecialCharTok{\textasciitilde{}} \StringTok{"Inactive"}\NormalTok{, }
\NormalTok{    TotalSteps }\SpecialCharTok{\textgreater{}=} \DecValTok{5000} \SpecialCharTok{\&}\NormalTok{ TotalSteps }\SpecialCharTok{\textless{}} \DecValTok{8750} \SpecialCharTok{\textasciitilde{}} \StringTok{"Lightly Active"}\NormalTok{, }
\NormalTok{    TotalSteps }\SpecialCharTok{\textgreater{}=} \DecValTok{8750} \SpecialCharTok{\&}\NormalTok{ TotalSteps }\SpecialCharTok{\textless{}} \DecValTok{12500} \SpecialCharTok{\textasciitilde{}} \StringTok{"Fairly Active"}\NormalTok{, }
\NormalTok{    TotalSteps }\SpecialCharTok{\textgreater{}=} \DecValTok{12500} \SpecialCharTok{\textasciitilde{}} \StringTok{"Very Active"}
\NormalTok{  ), }\AttributeTok{SleepSufficiency =} \FunctionTok{case\_when}\NormalTok{(}
\NormalTok{    TotalMinutesAsleep }\SpecialCharTok{\textless{}} \DecValTok{420} \SpecialCharTok{\textasciitilde{}} \StringTok{"Not Enough"}\NormalTok{,}
\NormalTok{    TotalMinutesAsleep }\SpecialCharTok{\textgreater{}=} \DecValTok{420} \SpecialCharTok{\textasciitilde{}} \StringTok{"Enough"}
\NormalTok{  ))}


\CommentTok{\#Analyzing{-} correlation b/w SedentaryMinutes and TotalMinutesAsleep}
\FunctionTok{cor}\NormalTok{(sleep\_steps}\SpecialCharTok{$}\NormalTok{SedentaryMinutes, sleep\_steps}\SpecialCharTok{$}\NormalTok{TotalMinutesAsleep) }\CommentTok{\#{-}0.599394}
\end{Highlighting}
\end{Shaded}

\begin{verbatim}
## [1] -0.599394
\end{verbatim}

The correlation between SedentaryMinutes and TotalMinutesAsleep is
-0.599394 or -0.6, which is considered to be a moderately strong
negative correlation. {[}go to Figure 5{]}

\begin{Shaded}
\begin{Highlighting}[]
\CommentTok{\#Analyzing{-} counting how many observations in EnoughSleep}
\NormalTok{sleep\_steps }\SpecialCharTok{\%\textgreater{}\%} \FunctionTok{count}\NormalTok{(SleepSufficiency) }\CommentTok{\#Enough{-} 231 observations; Not Enough{-} 182 observations}
\end{Highlighting}
\end{Shaded}

\begin{verbatim}
##   SleepSufficiency   n
## 1           Enough 231
## 2       Not Enough 182
\end{verbatim}

There are 231 observations (55.9\%) that were categorized as `Enough'
sleep, meaning users had at least 420 minutes of sleep that day. There
are 182 observations (44.1\%) that were categorized as `Not Enough'
sleep, meaning users had less than 420 minutes of sleep that day. This
result is relatively consistent with our previous analysis of healthy
sleep users and sleep minutes. {[}go to Figure 3{]}

\begin{Shaded}
\begin{Highlighting}[]
\CommentTok{\#Analyzing{-} creating new column to categorize the activity level (based on activity minutes)}
\FunctionTok{mean}\NormalTok{(sleep\_activity\_cleaned}\SpecialCharTok{$}\NormalTok{SedentaryMinutes) }\CommentTok{\# 712.1695}
\end{Highlighting}
\end{Shaded}

\begin{verbatim}
## [1] 712.1695
\end{verbatim}

\begin{Shaded}
\begin{Highlighting}[]
\FunctionTok{mean}\NormalTok{(sleep\_activity\_cleaned}\SpecialCharTok{$}\NormalTok{LightlyActiveMinutes) }\CommentTok{\# 216.8547}
\end{Highlighting}
\end{Shaded}

\begin{verbatim}
## [1] 216.8547
\end{verbatim}

\begin{Shaded}
\begin{Highlighting}[]
\FunctionTok{mean}\NormalTok{(sleep\_activity\_cleaned}\SpecialCharTok{$}\NormalTok{FairlyActiveMinutes) }\CommentTok{\# 18.03874}
\end{Highlighting}
\end{Shaded}

\begin{verbatim}
## [1] 18.03874
\end{verbatim}

\begin{Shaded}
\begin{Highlighting}[]
\FunctionTok{mean}\NormalTok{(sleep\_activity\_cleaned}\SpecialCharTok{$}\NormalTok{VeryActiveMinutes) }\CommentTok{\# 25.18886}
\end{Highlighting}
\end{Shaded}

\begin{verbatim}
## [1] 25.18886
\end{verbatim}

\begin{Shaded}
\begin{Highlighting}[]
\NormalTok{sleep\_activity\_mins }\OtherTok{\textless{}{-}}\NormalTok{ sleep\_steps }\SpecialCharTok{\%\textgreater{}\%} 
  \FunctionTok{group\_by}\NormalTok{(Id) }\SpecialCharTok{\%\textgreater{}\%}
  \FunctionTok{mutate}\NormalTok{(}\AttributeTok{AwakeInBed =}\NormalTok{ TotalTimeInBed }\SpecialCharTok{{-}}\NormalTok{ TotalMinutesAsleep, }
         \AttributeTok{ActivityLevelMins =} \FunctionTok{case\_when}\NormalTok{(}
\NormalTok{           SedentaryMinutes }\SpecialCharTok{\textgreater{}} \FloatTok{712.1965} \SpecialCharTok{\&}
\NormalTok{             LightlyActiveMinutes }\SpecialCharTok{\textless{}} \FloatTok{216.8547} \SpecialCharTok{\&} 
\NormalTok{             FairlyActiveMinutes }\SpecialCharTok{\textless{}} \FloatTok{18.03874} \SpecialCharTok{\&}
\NormalTok{             VeryActiveMinutes }\SpecialCharTok{\textless{}} \FloatTok{25.18886} \SpecialCharTok{\textasciitilde{}} \StringTok{"Inactive"}\NormalTok{,}
\NormalTok{           SedentaryMinutes }\SpecialCharTok{\textless{}} \FloatTok{712.1965} \SpecialCharTok{\&}
\NormalTok{             LightlyActiveMinutes }\SpecialCharTok{\textgreater{}} \FloatTok{216.8547} \SpecialCharTok{\&} 
\NormalTok{             FairlyActiveMinutes }\SpecialCharTok{\textless{}} \FloatTok{18.03874} \SpecialCharTok{\&}
\NormalTok{             VeryActiveMinutes }\SpecialCharTok{\textless{}} \FloatTok{25.18886} \SpecialCharTok{\textasciitilde{}} \StringTok{"Lightly Active"}\NormalTok{,}
\NormalTok{           SedentaryMinutes }\SpecialCharTok{\textless{}} \FloatTok{712.1965} \SpecialCharTok{\&}
\NormalTok{             LightlyActiveMinutes }\SpecialCharTok{\textless{}} \FloatTok{216.8547} \SpecialCharTok{\&} 
\NormalTok{             FairlyActiveMinutes }\SpecialCharTok{\textgreater{}} \FloatTok{18.03874} \SpecialCharTok{\&}
\NormalTok{             VeryActiveMinutes }\SpecialCharTok{\textless{}} \FloatTok{25.18886} \SpecialCharTok{\textasciitilde{}} \StringTok{"Fairly Active"}\NormalTok{,}
\NormalTok{           SedentaryMinutes }\SpecialCharTok{\textless{}} \FloatTok{712.1965} \SpecialCharTok{\&}
\NormalTok{             LightlyActiveMinutes }\SpecialCharTok{\textless{}} \FloatTok{216.8547} \SpecialCharTok{\&}
\NormalTok{             FairlyActiveMinutes }\SpecialCharTok{\textless{}} \FloatTok{18.03874} \SpecialCharTok{\&} 
\NormalTok{             VeryActiveMinutes }\SpecialCharTok{\textgreater{}} \FloatTok{25.18886} \SpecialCharTok{\textasciitilde{}} \StringTok{"Very Active"}\NormalTok{), }
         \AttributeTok{SleepSufficiency =} \FunctionTok{case\_when}\NormalTok{(}
\NormalTok{           TotalMinutesAsleep }\SpecialCharTok{\textless{}} \DecValTok{420} \SpecialCharTok{\textasciitilde{}} \StringTok{"Not Enough"}\NormalTok{,}
\NormalTok{           TotalMinutesAsleep }\SpecialCharTok{\textgreater{}=} \DecValTok{420} \SpecialCharTok{\textasciitilde{}} \StringTok{"Enough"}\NormalTok{)) }\SpecialCharTok{\%\textgreater{}\%} 
  \FunctionTok{drop\_na}\NormalTok{()}


\CommentTok{\#Analyzing{-} correlation b/w AwakeInBed and TotalMinutesAsleep}
\FunctionTok{cor}\NormalTok{(sleep\_activity\_mins}\SpecialCharTok{$}\NormalTok{AwakeInBed, sleep\_activity\_mins}\SpecialCharTok{$}\NormalTok{TotalMinutesAsleep) }\CommentTok{\#{-}0.1385993}
\end{Highlighting}
\end{Shaded}

\begin{verbatim}
## [1] -0.1385993
\end{verbatim}

\begin{Shaded}
\begin{Highlighting}[]
\CommentTok{\#On average, how long does it take users to sleep?}
\FunctionTok{mean}\NormalTok{(sleep\_activity\_mins}\SpecialCharTok{$}\NormalTok{AwakeInBed) }\CommentTok{\#38.83 mins}
\end{Highlighting}
\end{Shaded}

\begin{verbatim}
## [1] 38.82993
\end{verbatim}

\emph{{Note: The number of rows in this data has dropped in comparison
to sleep\_steps data because some observations had ambiguous results and
did not fit the case criteria. }}

The correlation between AwakeInBed and TotalMinutesAsleep is -0.1385993
or -0.14, which is considered a weak negative correlation. It is not
correlated enough to infer a strong relationship between those two
variables. {[}go to Figure 6{]}

On average, it takes users 38.83 minutes to fall asleep. That is higher
than the healthy recommended level of 10 to 20 minutes
\href{https://www.healthline.com/health/healthy-sleep/how-long-does-it-take-to-fall-asleep\#:~:text=It\%20should\%20take\%20between\%2010,fall\%20asleep\%20much\%20more\%20quickly}{4}.
Users are having a harder time falling asleep. This could be due to
other factors such as screen time before bed, stress, user's health or
medical records, etc.

\hypertarget{share}{%
\section{Share}\label{share}}

Using data visualization to show trends and relationships amongst
variables.

Aims to answer:

\begin{itemize}
\tightlist
\item
  Does regular physical activity lead to better sleep?
\item
  How does being active vs being sedentary impact my sleeping schedule?
  Is there a relationship between sleep and activeness?
\item
  Do users spend a considerable amount of time being sedentary?
\item
  How does this translate in terms of improving Bellabeat?
\end{itemize}

\hypertarget{code-2}{%
\subsection{Code}\label{code-2}}

\hypertarget{loading-package}{%
\subsubsection{Loading Package}\label{loading-package}}

\begin{Shaded}
\begin{Highlighting}[]
\CommentTok{\#Visualization{-} look at relationship b/w total steps and calories}
\FunctionTok{library}\NormalTok{(ggplot2)}
\end{Highlighting}
\end{Shaded}

\begin{verbatim}
## Warning: package 'ggplot2' was built under R version 4.1.3
\end{verbatim}

\hypertarget{visualizing-data}{%
\subsubsection{Visualizing Data}\label{visualizing-data}}

\hypertarget{figure-1-number-of-observations-by-activity-level-mins.}{%
\paragraph{Figure 1: Number of Observations by Activity Level
(Mins.)}\label{figure-1-number-of-observations-by-activity-level-mins.}}

\textbf{Who is our target audience (the highest number of users) based
on activity level (Minutes)?}

\begin{Shaded}
\begin{Highlighting}[]
\FunctionTok{ggplot}\NormalTok{(}\AttributeTok{data =}\NormalTok{ sleep\_activity\_mins, }\FunctionTok{aes}\NormalTok{(ActivityLevelMins)) }\SpecialCharTok{+} 
  \FunctionTok{geom\_bar}\NormalTok{(}\FunctionTok{aes}\NormalTok{(}\AttributeTok{fill =}\NormalTok{ ActivityLevelMins)) }\SpecialCharTok{+}
  \FunctionTok{geom\_text}\NormalTok{(}\AttributeTok{stat =} \StringTok{\textquotesingle{}count\textquotesingle{}}\NormalTok{, }\FunctionTok{aes}\NormalTok{(}\AttributeTok{label =}\NormalTok{ ..count..), }\AttributeTok{vjust =}  \SpecialCharTok{{-}}\FloatTok{0.5}\NormalTok{) }\SpecialCharTok{+}
  \FunctionTok{labs}\NormalTok{(}\AttributeTok{title =} \StringTok{\textquotesingle{}Number of Observations by Activity Level (Mins)\textquotesingle{}}\NormalTok{)}
\end{Highlighting}
\end{Shaded}

\includegraphics{Bellabeat-Data-Analysis-with-R_files/figure-latex/ggplot-1.pdf}

Most observations are from users that spend the most time being inactive
or sedentary (i.e.~sitting, lying down) and lightly active
(i.e.~standing, doing household chores, walking).

These are users that wear and use Bellabeat products the most. If the
marketing team wants to bring in more users, they can target these niche
audiences.

\hypertarget{figure-2-total-active-minutes-vs.-calories-by-activity-level-steps}{%
\paragraph{Figure 2: Total Active Minutes vs.~Calories by Activity Level
(Steps)}\label{figure-2-total-active-minutes-vs.-calories-by-activity-level-steps}}

\begin{Shaded}
\begin{Highlighting}[]
\CommentTok{\#Total Active Minutes vs Calories by Activity Level (Steps)}
\FunctionTok{ggplot}\NormalTok{(sleep\_steps, }\AttributeTok{mapping =} \FunctionTok{aes}\NormalTok{(}
  \AttributeTok{x =}\NormalTok{ TotalActiveMinutes, }\AttributeTok{y =}\NormalTok{ Calories, }\AttributeTok{fill =}\NormalTok{ StepsActivityLevel, }\AttributeTok{color =}\NormalTok{ StepsActivityLevel)) }\SpecialCharTok{+} 
  \FunctionTok{geom\_point}\NormalTok{() }\SpecialCharTok{+}
  \FunctionTok{labs}\NormalTok{(}\AttributeTok{title =} \StringTok{"Total Active Minutes vs Calories by Activity Level (Steps)"}\NormalTok{)}
\end{Highlighting}
\end{Shaded}

\includegraphics{Bellabeat-Data-Analysis-with-R_files/figure-latex/unnamed-chunk-18-1.pdf}

This graph looks at the total active minutes (Sedentary Minutes +
LightlyActiveMinutes + FairlyActiveMinutes + VeryActiveMinutes) and the
calories burned at each activity level.

There is a positive relationship between TotalActiveMinutes and
Calories. The more time users are active, the more calories users burn.
The activity level data is consistent with this analysis. Users who are
very active spend the most time being active and burn the most calories
per day; users who are inactive spend most of their time being sedentary
and burn the least calories per day.

\hypertarget{figure-3-percentage-of-sleep-sufficient-observations}{%
\paragraph{Figure 3: Percentage of Sleep Sufficient
Observations}\label{figure-3-percentage-of-sleep-sufficient-observations}}

\textbf{Are users getting enough sleep?}

\begin{Shaded}
\begin{Highlighting}[]
\CommentTok{\#creating a df to get an aggregate of observations (n) and the percentage for sleep sufficiency}
\NormalTok{pie\_chart }\OtherTok{\textless{}{-}}\NormalTok{ sleep\_steps }\SpecialCharTok{\%\textgreater{}\%} 
  \FunctionTok{group\_by}\NormalTok{(SleepSufficiency) }\SpecialCharTok{\%\textgreater{}\%} 
  \FunctionTok{count}\NormalTok{() }\SpecialCharTok{\%\textgreater{}\%} 
  \FunctionTok{ungroup}\NormalTok{() }\SpecialCharTok{\%\textgreater{}\%} 
  \FunctionTok{mutate}\NormalTok{(}\AttributeTok{percent =}\NormalTok{ n }\SpecialCharTok{/} \FunctionTok{sum}\NormalTok{(n)) }\SpecialCharTok{\%\textgreater{}\%} 
  \FunctionTok{arrange}\NormalTok{(percent) }\SpecialCharTok{\%\textgreater{}\%} 
  \FunctionTok{mutate}\NormalTok{(}\AttributeTok{labels =}\NormalTok{ scales}\SpecialCharTok{::}\FunctionTok{percent}\NormalTok{(percent))}

\CommentTok{\#making a pie chart with annotations}
\FunctionTok{ggplot}\NormalTok{(pie\_chart, }\FunctionTok{aes}\NormalTok{(}\AttributeTok{x =} \StringTok{""}\NormalTok{, }\AttributeTok{y =}\NormalTok{ percent, }\AttributeTok{fill =}\NormalTok{ SleepSufficiency)) }\SpecialCharTok{+}
  \FunctionTok{geom\_col}\NormalTok{() }\SpecialCharTok{+}
  \FunctionTok{geom\_label}\NormalTok{(}\FunctionTok{aes}\NormalTok{(}\AttributeTok{label =}\NormalTok{ labels), }\AttributeTok{position =} \FunctionTok{position\_stack}\NormalTok{(}\AttributeTok{vjust =} \FloatTok{0.5}\NormalTok{), }\AttributeTok{show.legend =} \ConstantTok{FALSE}\NormalTok{) }\SpecialCharTok{+}
  \FunctionTok{geom\_text}\NormalTok{(}\FunctionTok{aes}\NormalTok{(}\AttributeTok{label =}\NormalTok{ labels), }\AttributeTok{position =} \FunctionTok{position\_stack}\NormalTok{(}\AttributeTok{vjust =} \FloatTok{0.5}\NormalTok{), }\AttributeTok{show.legend =} \ConstantTok{FALSE}\NormalTok{) }\SpecialCharTok{+}
  \FunctionTok{coord\_polar}\NormalTok{(}\AttributeTok{theta =} \StringTok{"y"}\NormalTok{) }\SpecialCharTok{+} 
  \FunctionTok{theme\_void}\NormalTok{() }\SpecialCharTok{+}
  \FunctionTok{labs}\NormalTok{(}\AttributeTok{title =} \StringTok{"Percentage of Sleep Sufficient Observations"}\NormalTok{)}
\end{Highlighting}
\end{Shaded}

\includegraphics{Bellabeat-Data-Analysis-with-R_files/figure-latex/unnamed-chunk-19-1.pdf}

This visualization is consistent with our previous analysis. More than
half of the users (56\%) get enough sleep, meaning at least 420 minutes
or 7 hours a day.

\hypertarget{figure-4-comparing-user-sleep-sufficient-observations-by-activity-level-minutes}{%
\paragraph{Figure 4: Comparing User Sleep Sufficient Observations by
Activity Level
(Minutes)}\label{figure-4-comparing-user-sleep-sufficient-observations-by-activity-level-minutes}}

\begin{Shaded}
\begin{Highlighting}[]
\NormalTok{sleep\_activity\_mins }\SpecialCharTok{\%\textgreater{}\%} 
  \FunctionTok{group\_by}\NormalTok{(SleepSufficiency) }\SpecialCharTok{\%\textgreater{}\%} 
  \FunctionTok{count}\NormalTok{(ActivityLevelMins) }\SpecialCharTok{\%\textgreater{}\%} 
  \FunctionTok{ggplot}\NormalTok{(sleep\_activity2, }\AttributeTok{mapping =} \FunctionTok{aes}\NormalTok{(}\AttributeTok{x =}\NormalTok{ ActivityLevelMins, }\AttributeTok{y =}\NormalTok{ n)) }\SpecialCharTok{+} 
  \FunctionTok{geom\_bar}\NormalTok{(}\FunctionTok{aes}\NormalTok{(}\AttributeTok{fill =}\NormalTok{ SleepSufficiency), }
           \AttributeTok{stat =} \StringTok{"identity"}\NormalTok{, }\AttributeTok{color =} \StringTok{"white"}\NormalTok{, }
           \AttributeTok{position =} \FunctionTok{position\_dodge}\NormalTok{(}\FloatTok{0.9}\NormalTok{)) }\SpecialCharTok{+} 
  \FunctionTok{labs}\NormalTok{(}\AttributeTok{title =} \StringTok{"Comparing User Sleep Sufficient Observations by Activity Level (Mins)"}\NormalTok{)}
\end{Highlighting}
\end{Shaded}

\includegraphics{Bellabeat-Data-Analysis-with-R_files/figure-latex/unnamed-chunk-20-1.pdf}

There is a clear trend; As user activity level decreases, users are less
likely to get sufficient sleep (at least 420 minutes or 7 hours a
night).

Poor sleep may contribute to physical inactivity
\href{https://journals.sagepub.com/doi/abs/10.1177/1559827614544437}{5}.
The group that are least likely to get enough sleep is the sedentary
group. While many sedentary users have enough sleep, there is still a
high number of `Not Enough' sleep observations compared to other
activity levels.

\hypertarget{figure-5-sedentary-minutes-vs.-total-minutes-asleep}{%
\paragraph{Figure 5: Sedentary Minutes vs.~Total Minutes
Asleep}\label{figure-5-sedentary-minutes-vs.-total-minutes-asleep}}

\textbf{How does being sedentary impact sleep?}

\begin{Shaded}
\begin{Highlighting}[]
\FunctionTok{ggplot}\NormalTok{(sleep\_activity\_cleaned, }\AttributeTok{mapping =} \FunctionTok{aes}\NormalTok{(}\AttributeTok{x =}\NormalTok{ SedentaryMinutes, }\AttributeTok{y =}\NormalTok{ TotalMinutesAsleep)) }\SpecialCharTok{+} \FunctionTok{geom\_point}\NormalTok{() }\SpecialCharTok{+} \FunctionTok{geom\_smooth}\NormalTok{() }\SpecialCharTok{+} \FunctionTok{labs}\NormalTok{(}\AttributeTok{title =} \StringTok{"Sedentary Minutes vs Total Minutes Asleep"}\NormalTok{) }
\end{Highlighting}
\end{Shaded}

\begin{verbatim}
## `geom_smooth()` using method = 'loess' and formula 'y ~ x'
\end{verbatim}

\includegraphics{Bellabeat-Data-Analysis-with-R_files/figure-latex/unnamed-chunk-21-1.pdf}

It is evident that there is a negative relationship between
Sedentaryminutes and TotalMinutesAsleep. The more users stay sedentary,
the fewer users sleep.

This can be concerning for the user's health because Bellabeat users
spend a lot of time sedent (on average, 571.6 minutes or 9.52 hours a
day). Inactiveness can be strongly associated with insomnia and
restlessness. Increasing exercise or physical activity can improve sleep
duration and alertness in the daytime
\href{https://www.ncbi.nlm.nih.gov/pmc/articles/PMC6352043/}{6}.

In this study, researchers examined the association between sedentary
behaviours and sleep time and found that physical activity increased
with sleep duration in younger respondents (aged 20-39)
\href{https://www.sciencedirect.com/science/article/abs/pii/S0091743514002035\%20→\%20physical\%20activity\%20and\%20sleep\%20duration}{7}.

\hypertarget{figure-6-awakeinbed-vs.-totalminutesasleep}{%
\paragraph{Figure 6: AwakeInBed
vs.~TotalMinutesAsleep}\label{figure-6-awakeinbed-vs.-totalminutesasleep}}

\textbf{How does being awake in bed relate to sleep?}

\begin{Shaded}
\begin{Highlighting}[]
\FunctionTok{ggplot}\NormalTok{(}\AttributeTok{data =}\NormalTok{ sleep\_activity\_mins, }\AttributeTok{mapping =} \FunctionTok{aes}\NormalTok{(}\AttributeTok{x =}\NormalTok{ AwakeInBed, }\AttributeTok{y =}\NormalTok{ TotalMinutesAsleep)) }\SpecialCharTok{+} 
  \FunctionTok{geom\_point}\NormalTok{() }\SpecialCharTok{+} 
  \FunctionTok{geom\_smooth}\NormalTok{() }\SpecialCharTok{+}
  \FunctionTok{labs}\NormalTok{(}\AttributeTok{title =} \StringTok{"Awake in Bed (Mins) vs Total Minutes Asleep"}\NormalTok{)}
\end{Highlighting}
\end{Shaded}

\begin{verbatim}
## `geom_smooth()` using method = 'loess' and formula 'y ~ x'
\end{verbatim}

\includegraphics{Bellabeat-Data-Analysis-with-R_files/figure-latex/unnamed-chunk-22-1.pdf}

This graph shows a decreasing concavity, showing a negative relationship
between AwakeInBed and TotalMinutesAsleep. The longer users stay awake
in bed, the fewer users sleep. The correlation between these two
variables shows a weak relationship.

\hypertarget{act}{%
\section{Act}\label{act}}

\hypertarget{limitations-1}{%
\subsection{Limitations}\label{limitations-1}}

If the dataset included demographic data, we can analyze the differences
in sleep and activity by age. Age can affect sleep and activity, so
having that extra data can provide more additional and insightful
analysis.

\hypertarget{key-findings}{%
\subsection{Key Findings}\label{key-findings}}

\begin{itemize}
\tightlist
\item
  On average, users stay sedentary for 717 minutes per day.
\item
  On average, users take 8541 steps per day which are lower than
  recommended (10k steps/day).
\item
  21\% of users are active and take at least 10,000 steps per day.
\item
  21\% of users are inactive and take less than 5,000 steps per day.
\item
  On average, users sleep for 377.6 minutes per night.
\item
  50\% of users get at least 420 minutes of sleep per night.
\item
  There is a strong negative correlation between SedentaryMinutes and
  TotalMinutesAsleep.
\item
  On average, it takes users 38.83 minutes to fall asleep.
\item
  Most of the users are between inactive and lightly active.
\item
  The more time users stay active, the more calories they burn.
\item
  More than half of the users (56\%) get sufficient sleep (at least 420
  minutes).
\item
  As user activity level decreases, users are less likely to get
  sufficient sleep.
\item
  The longer users stay awake in bed, the fewer users sleep.
\end{itemize}

\hypertarget{recommendations}{%
\subsection{Recommendations}\label{recommendations}}

\hypertarget{upgrades}{%
\subsubsection{Upgrades}\label{upgrades}}

==\textgreater{} \textbf{Improve machine learning.}

\begin{itemize}
\tightlist
\item
  Continue to improve on collecting accurate data to encourage users to
  wear their Bellabeat tracker more and take control of their own
  fitness and health journey.
\end{itemize}

==\textgreater{} \textbf{Improve wearability}

\begin{itemize}
\tightlist
\item
  Ensure the Bellabeat tracker is comfortable while staying trendy to
  wear for users to encourage logging participation.
\end{itemize}

\hypertarget{new-features}{%
\subsubsection{New Features}\label{new-features}}

==\textgreater{} \textbf{Allow users to set up a wake-up and sleep buzz
alert.}

\begin{itemize}
\tightlist
\item
  Sending a reminder for bedtime will encourage more sleep by reducing
  screen time. According to a poll, ``95\% of people said they regularly
  use some type of electronics within an hour of bedtime.''
  \href{https://www.sleepfoundation.org/bedroom-environment/technology-in-the-bedroom}{8}
  We can get lost in our electronics at night before bed and feel not
  tired because we're stimulated.
\end{itemize}

==\textgreater{} \textbf{Meditation, breathing, and other mindful
exercises.}

\begin{itemize}
\tightlist
\item
  Meditation and breathing exercises can improve the quality of your
  sleep. It can help users who are sleeping less or have trouble
  sleeping relax and bring calmness
  \href{https://www.sleepfoundation.org/insomnia/treatment/meditation}{9}.
  This new feature in the Time Watch will incorporate guided breathing
  exercises that users can pick from throughout the day.
\end{itemize}

\hypertarget{user-growth}{%
\paragraph{User Growth}\label{user-growth}}

==\textgreater{} \textbf{Connect to users' medical records.}

\begin{itemize}
\tightlist
\item
  Bellabeat can personalize each user's health journey by connecting to
  their medical records. That way, doctors or providers can better track
  your health and self-manage their health. Bellabeat can expand in the
  healthcare industry by working with healthcare insurance companies to
  use Bellabeat for patients that need regular fitness tracking
  \href{https://www.ncbi.nlm.nih.gov/pmc/articles/PMC6746089/}{10}.
\end{itemize}

==\textgreater{} \textbf{Rewards point system.}

\begin{itemize}
\tightlist
\item
  Bellabeat can create a rewards point system for unlocking
  achievements, upholding streaks, or even completing a fun fitness
  challenge. As users gather points, they can trade them in for rewards
  such as free workout programs, deals with active equipment, and much
  more.
\end{itemize}

\hypertarget{next-up}{%
\subsection{Next Up\ldots{}}\label{next-up}}

There is additional data in the dataset to analyze to provide more
extensive insights such as users' logs on weight, intensities, heart
rate, and METs.

\hypertarget{sources}{%
\section{Sources}\label{sources}}

{[}1{]}
\href{https://www.medicinenet.com/how_many_steps_a_day_is_considered_active/article.htm}{How
Many Steps a Day Is Considered Active?}

{[}2{]}
\href{https://www.sleepfoundation.org/sleep-hygiene/what-is-healthy-sleep}{Healthy
Sleep: What Is It and Are You Getting It?}

{[}3{]}
\href{https://ijbnpa.biomedcentral.com/articles/10.1186/1479-5868-8-79\#:~:text=In\%20summary\%2C\%20at\%20least\%20in,populace\%20\%5B3\%2C\%2023\%5D}{How
many steps/day are enough? for adults - International Journal of
Behavioral Nutrition and Physical Activity}

{[}4{]}
\href{https://www.healthline.com/health/healthy-sleep/how-long-does-it-take-to-fall-asleep\#:~:text=It\%20should\%20take\%20between\%2010,fall\%20asleep\%20much\%20more\%20quickly}{How
Long Does It Take to Fall Asleep? Average Time and Tips}

{[}5{]}
\href{https://journals.sagepub.com/doi/abs/10.1177/1559827614544437}{Bidirectional
Relationship Between Exercise and Sleep: Implications for Exercise
Adherence and Sleep Improvement - Christopher E. Kline, 2014}

{[}6{]} \href{https://www.ncbi.nlm.nih.gov/pmc/articles/PMC6352043/}{Are
Sedentary Behaviors Associated with Sleep Duration? A Cross-Sectional
Case from Croatia}

{[}7{]}
\href{https://www.sciencedirect.com/science/article/abs/pii/S0091743514002035}{Associations
between physical activity, sedentary time, sleep duration and daytime
sleepiness in US adults}

{[}8{]}
\href{https://www.sleepfoundation.org/bedroom-environment/technology-in-the-bedroom}{Technology
in the Bedroom}

{[}9{]}
\href{https://www.sleepfoundation.org/insomnia/treatment/meditation}{Can
Meditation Treat Insomnia}

{[}10{]}
\href{https://www.ncbi.nlm.nih.gov/pmc/articles/PMC6746089/}{Wearable
Health Technology and Electronic Health Record Integration: Scoping
Review and Future Directions}

\end{document}
